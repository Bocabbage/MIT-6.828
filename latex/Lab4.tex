\documentclass[12pt, letterpaper]{report}
\usepackage{enumitem}
\usepackage[a4paper, margin=0.7in, top=20mm, bottom=20mm]{geometry}
\usepackage{mathspec}
\usepackage[UTF8]{ctex}
\usepackage[colorlinks=true, linkcolor=black, urlcolor=blue]{hyperref}
\usepackage{listings}
\usepackage{xcolor}
\usepackage{graphicx}
\usepackage{xparse}
\usepackage{tcolorbox}
\usepackage{mdframed}

\graphicspath{ {./pics/} }

% ---- Code Section Style ----
\colorlet{mygray}{black!30}
\colorlet{mygreen}{green!60!blue}
\colorlet{mymauve}{red!60!blue}

\tcbuselibrary{breakable}
\NewDocumentCommand{\exercise}{ m +m }{
    {
        \edef\originalParIndent{\the\parindent}
        \begin{tcolorbox}[breakable,arc=0mm,boxrule=0.8pt]
            \setlength{\parindent}{\originalParIndent}
            \noindent
            \textbf{\large Exercise #1}
            \indent
            #2
        \end{tcolorbox}
    }
}

\tcbuselibrary{breakable}
\NewDocumentCommand{\question}{ m +m }{
    {
        \edef\originalParIndent{\the\parindent}
        \begin{tcolorbox}[breakable,arc=0mm,boxrule=0.8pt]
            \setlength{\parindent}{\originalParIndent}
            \noindent
            \textbf{\large Question}
            \indent
            #2
        \end{tcolorbox}
    }
}

\tcbuselibrary{breakable}
\NewDocumentCommand{\challenge}{ m +m }{
    {
        \edef\originalParIndent{\the\parindent}
        \begin{tcolorbox}[breakable,arc=0mm,boxrule=0.8pt]
            \setlength{\parindent}{\originalParIndent}
            \noindent
            \textbf{\large Challenge!}
            \indent
            #2
        \end{tcolorbox}
    }
}

\lstdefinelanguage
   [x64]{Assembler}     % add a "x64" dialect of Assembler
   [x86masm]{Assembler} % based on the "x86masm" dialect
   % with these extra keywords:
   {morekeywords={CDQE,CQO,CMPSQ,CMPXCHG16B,JRCXZ,LODSQ,MOVSXD, %
                  POPFQ,PUSHFQ,SCASQ,STOSQ,IRETQ,RDTSCP,SWAPGS, %
                  rax,rdx,rcx,rbx,rsi,rdi,rsp,rbp, %
                  r8,r8d,r8w,r8b,r9,r9d,r9w,r9b, %
                  r10,r10d,r10w,r10b,r11,r11d,r11w,r11b, %
                  r12,r12d,r12w,r12b,r13,r13d,r13w,r13b, %
                  r14,r14d,r14w,r14b,r15,r15d,r15w,r15b}} % etc.


\lstset{
  backgroundcolor=\color{gray!10},  
  basicstyle=\ttfamily,
  columns=fullflexible,
  breakatwhitespace=false,      
  breaklines=true,                
  captionpos=b,                    
  commentstyle=\color{mygreen}, 
  extendedchars=true,              
  frame=single,                   
  keepspaces=true,             
  keywordstyle=\color{blue},                    
  numbers=none,                
  numbersep=5pt,                   
  numberstyle=\tiny\color{blue}, 
  rulecolor=\color{mygray},        
  showspaces=false,               
  showtabs=false,                 
  stepnumber=5,                  
  stringstyle=\color{mymauve},    
  tabsize=3,                      
  title=\lstname                
}

\lstdefinestyle{CStyle}{  
    % commentstyle=\color{mGreen},
    % keywordstyle=\color{magenta},
    % numberstyle=\tiny\color{mGray},
    % stringstyle=\color{mPurple},
    basicstyle=\footnotesize,
    breakatwhitespace=false,                              
    showstringspaces=false,
    language=C
}

\lstdefinestyle{AssemblyStyle}{  
    basicstyle=\footnotesize,
    breakatwhitespace=false,                              
    showstringspaces=false,
    language=[x64] Assembler
}


\lstdefinestyle{MakeFileStyle}{  
    basicstyle=\footnotesize,
    breakatwhitespace=false,                              
    showstringspaces=false,
    language=[gnu] make
}


\setitemize[1]{itemsep=0pt,partopsep=0pt,parsep=\parskip,topsep=5pt}

% ----------------------------


\setcounter{chapter}{0}
\setlength{\parindent}{2em}
\setmainfont{Times New Roman}
\setcounter{tocdepth}{1}
\setcounter{secnumdepth}{1}

\title{MIT6.828 Lab4: Preemptive Multitasking}
\author{Zhuofan Zhang}
\date{April 2020}



\begin{document}

\maketitle
% ---- Contents ----
\pagenumbering{roman}
\renewcommand\contentsname{\Huge Contents}
\tableofcontents{}
% ------------------

\newpage
\pagenumbering{arabic}

% ---- part A ----
\chapter[\Large Multiprocessor Support and Cooperative Multitasking]{Multiprocessor Support and Cooperative Multitasking}
\section[\large Multiprocessor Support]{Multiprocessor Support}
本次Lab的内容是对JOS进行补充以提供多处理器支持,并实现任务调度功能。\par

JOS实现的多核支持属于\textsl{对称多核支持(Symmertric Multiprocessing, SMP)},即所有处理器
对资源(内存、IO总线等)的访问是平等的。SMP的概念是在系统初始化完成后建立的,
在Boot阶段,仍然需要选择一个CPU作为\textsl{启动处理器(Bootstrap Processor, BSP)},用来初始化
资源并启动OS,最后启动其他的CPU(称为\textsl{应用处理器(Application Processor, AP)})。
对BSP的选择是由BIOS完成的,在当前实验阶段,我们相当于完成了BSP的启动内容。\par 

在SMP体系中,每个CPU都拥有一个称为LAPIC(local Advanced-Programmable Interrupt Controller)的可编程中断单元,用于在系统中
传递中断信息,并提供CPU的唯一识别信息。因此,与多个CPU的通讯依赖于LAPIC单元。 \par

在本次实验中我们使用了 LAPIC 的如下功能:
\begin{itemize}
    \item[·]
    代码可以利用APIC-id确认自己运行于哪一个cpu上(见 cpunum() 的实现)
    \item[·]
    利用BSP来启动其他的AP(见 lapic\_startap() 的实现)
    \item[·]
    使用 LAPIC 中的时钟中断来实现抢断式调度(PartC,apic\_init())
\end{itemize}

CPU对LAPIC单元的访问使用的是一段映射到PAS特定位置的区域:memory-mapped I/O (MMIO)。MMIO是一段
硬连接到部分IO设备寄存器的物理地址区域,常用于访问设备的寄存器。JOS的VAS在MMIOBASE处
留有4MB的空间用于映射这段物理地址区域。\par
\newpage
\exercise{1}{
        \par 
        {
            Implement mmio\_map\_region in kern/pmap.c. 
            To see how this is used, look at the beginning of lapic\_init in kern/lapic.c. 
            You'll have to do the next exercise, too, 
            before the tests for mmio\_map\_region will run.
        } \par
}
\quad \par
这一个Exercise要求我们实现mmio\_map\_region()。我们查看该函数的注释可知,它就是用来
实现上文提到的MMIO在VAS中映射的函数。JOS的VAS中预留了 [MMIOBASE, MMIOLIM) 可供使用。
根据提示使用 boot\_map\_region() 并补上禁用缓存的PWT-flag,实现函数如下:\par 
\lstset{style=CStyle}
\setmainfont{Consolas}
\begin{lstlisting}
void *
mmio_map_region(physaddr_t pa, size_t size)
{
    static uintptr_t base = MMIOBASE;

    size = ROUNDUP(size, PGSIZE);
    if(base + size > MMIOLIM)
        panic("mmio_map_region: MMIOLIM overflow.\n");
    boot_map_region(kern_pgdir, base, size, pa, PTE_PCD | PTE_PWT);
    base += size;
    return (void *)(base - size);
    
}
\end{lstlisting}
\setmainfont{Times New Roman}

\section[\large Application Processor Bootstrap]{Application Processor Bootstrap}
在启动APs前,先启动的BSP必须能够获取APs的信息,如CPU数量、APIC IDs等。
这些信息被保留在BIOS中的 MP config table 中,kern/mpconfig.c/mp\_init() 
将其读出。\par 
kern/init.c/boot\_aps() 是整个流程的核心函数,它将AP的入口程序加载到内存,
并发送启动信号。AP的入口程序被加载到任意可用的低位640k地址(JOS将其加载到MPENTRY\_PADDR),
它与BSP的入口程序有一定的区别。 \par 
boot\_aps() 函数通过给每个AP的LAPIC发送STARTUP信号的方式唤醒它们,同时设置它们的entry位置,
即为AP准备的入口程序(位于MPENTRY\_PADDR),执行完入口程序后AP将运行mp\_main();与此同时,
BSP唤醒每一个AP时等待它的CPU状态被设置为CPU\_STARTED——即该AP成功启动后,再开始唤醒下一个AP。\par 

\newpage
\exercise{2}{
        \par 
        {
            Read boot\_aps() and mp\_main() in kern/init.c, 
            and the assembly code in kern/mpentry.S. 
            Make sure you understand the control flow transfer 
            during the bootstrap of APs. 
            Then modify your implementation of page\_init() 
            in kern/pmap.c to avoid adding the page at 
            MPENTRY\_PADDR to the free list, 
            so that we can safely copy and run AP bootstrap code 
            at that physical address. 
            Your code should pass the updated 
            check\_page\_free\_list() test 
            (but might fail the updated check\_kern\_pgdir() test, 
            which we will fix soon).
        } \par
}
\quad \par
这一个Exercise的编码任务比较简单:修改我们在Lab2实现的page\_init函数,将
现在存放有AP初始代码的那个物理页从空闲列表中拿出去:\par 
\lstset{style=CStyle}
\setmainfont{Consolas}
\begin{lstlisting}
void
page_init(void)
{
    // LAB 4:
    // Change your code to mark the physical page at MPENTRY_PADDR
    // as in use
    
    ...

    // 2) Base-memory
    size_t i;
    for (i = 1; i < npages_basemem; i++) {
        // Add for Lab4
        if(i * PGSIZE == MPENTRY_PADDR)
        {
            pages[i].pp_ref = 1;
            continue;
        }
            
        pages[i].pp_ref = 0;
        pages[i].pp_link = page_free_list;
        page_free_list = &pages[i];
    }

    ...

}
\end{lstlisting}
\setmainfont{Times New Roman}

\newpage
代码之外的任务是要求我们阅读 boot\_aps() 及 mp\_main() 源码以及
APs的entry汇编代码,捋清执行流。boot\_aps() 做的事情就是将mpentry.S搬运到前文提到的
MPENTRY\_PADDR物理页上,再循环唤醒APs(使用lapic\_startaps());APs被
唤醒后从mpentry.S开始执行,再跳转到mp\_main() 上。\par 
\quad \par 

\question{1}{
    {
        \par 
        Compare kern/mpentry.S side by side with boot/boot.S. 
        Bearing in mind that kern/mpentry.S is compiled and linked to 
        run above KERNBASE just like everything else in the kernel, 
        what is the purpose of macro MPBOOTPHYS? 
        Why is it necessary in kern/mpentry.S but not in boot/boot.S? 
        In other words, what could go wrong if it were omitted in kern/mpentry.S?
        Hint: recall the differences between the link address 
        and the load address that we have discussed in Lab 1.
    }
}

{\noindent AP的入口程序与BSP的入口程序差异主要表现在:}
\begin{itemize}
    \item[·]
    使用地址时需要使用MPBOOTPHYS-macro进行一步转换;
    \item[·]
    无需进行A20地址线的使能操作;
    \item[·]
    直接在入口处打开了分页功能,并使用已经被BSP设置好的内核页表目录
\end{itemize}
使用MPBOOTPHYS是因为mpentry.S被链接到了内核的高地址处(above KERNBASE),
需要进行转换。

\end{document}